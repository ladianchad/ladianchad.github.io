\CVSectionHeader{Projects}
\TitledListBegin{
  \TitleWithPeriod{\textbf{Floady : Autonomous Mobile Robot for Warehouses  - Floatic Co. South Korea} ~~ \href{https://www.youtube.com/watch?v=0acR1xwGQf8}{\textcolor{red}{\faYoutube} youtube}}{2022}[2024]
}
  \TitledListItem{Served as a core robotics engineer from the company’s initial phase, building successive production-grade robotics stacks for warehouse robots.}
  \TitledListItem{Custom-designed global planners, local controllers, and behavior trees in ROS2 Navigation2, with several improvements later integrated into the open-source project.}
  \TitledListItem{Developed full system integration, including sensor drivers (IMU, LiDAR, motor), Kafka/MQTT middleware for backend communication, and a Redis-based fleet supervision layer for robot-to-robot coordination. Explored large-scale simulation with Isaac Sim and AWS RoboMaker.}
  \TitledListItem{Deployed 10 robots simultaneously in a live warehouse, accumulating 100+ operating hours with ~90\% autonomous navigation success in dynamic human–robot conditions (planners at 10 Hz, controllers at 20 Hz, CPU 30-40\%).}
  \TitledListItem{Implemented an operator-friendly rollback routine for failed navigation recovery, raising the overall task success rate to >95\% and demonstrating a real-world testbed for scalable fleet supervision and decentralized decision-making.}
\TitledListEnd
\vspace{1em}
\TitledListBegin{
  \TitleWithPeriod{\textbf{Automotive marine rescue robot Capston Design - Hanyang University ERICA}}{2020}[2022]
}
  \TitledListItem{Co-developed a surface rescue robot, splitting computation between Jetson Nano (YOLOv5-based detection, 3–7 Hz) and Raspberry Pi (state estimation, control) for efficiency and cost-effectiveness.}
  \TitledListItem{Implemented Kalman and low-pass filtering for IMU/GPS fusion, enabling reliable estimation of pose and velocity under dynamic water conditions.}
  \TitledListItem{Achieved consistent performance in controlled pool environments (6–8 m detection range), while real-sea deployment highlighted the challenges of perception and robustness under tight budget and hardware constraints.}
  \TitledListItem{Experience revealed the difficulty of bringing academic methods to field-scale systems, motivating further research into robust perception and control under uncertainty.}
\TitledListEnd
\vspace{1em}
\TitledListBegin{
  \TitleWithPeriod{\textbf{LattePanda - Donation \& Recycling Robot, Hanyang University ERICA}  ~~ \href{https://www.youtube.com/watch?v=2ig5qAAA0DM&t=62s}{\textcolor{red}{\faYoutube} youtube}}{2020}
}
  \TitledListItem{Led system design of a mobile service robot that encouraged recycling and donations in public open spaces, combining navigation, perception, and human–robot interaction.}
  \TitledListItem{Implemented YOLOv4-based perception with SORT tracking to detect individuals carrying disposable cups and identify their engagement with the robot.}
  \TitledListItem{Developed lightweight obstacle avoidance by segmenting RealSense depth images and applying weighted heuristics for direction control.}
  \TitledListItem{Built ROS pipeline, motor control via PID, and PyQt/OpenCV UI that delivered engaging environmental animations and panda-robot gestures.}
  \TitledListItem{Project received strong recognition from faculty for its novel integration of perception, interaction, and social engagement.}
\TitledListEnd
\vspace{1em}
\TitledListBegin{
  \TitleWithPeriod{\textbf{Item Assembly AI Robot Challenge - Ministry of Science and ICT, Korea}  ~~ \href{https://www.youtube.com/watch?v=CTon4sS2iFs}{\textcolor{red}{\faYoutube} youtube} ~~ \href{https://www.newshyu.com/news/articleView.html?idxno=1001038}{\textcolor{gray}{\faGlobe} article}}{2020}[2021]
}
  \TitledListItem{Implemented the vision-based localization module for a furniture-assembly robotic arm in a national challenge jointly conducted by multiple university labs.}
  \TitledListItem{Transitioned from an initial RealSense depth-based approach to a custom pipeline with Basler GigE, OpenCV, triangulation, and camera calibration, gaining practical experience in robotic perception and optics.}
  \TitledListItem{Achieved real-time inference with localization accuracy within 20 mm, demonstrating applied vision research in constrained task environments.}
\TitledListEnd
\vspace{1em}
\TitledListBegin{
  \TitleWithPeriod{\textbf{Churo 2 - Outsourcing of Dog Care Robot, ROBOI Co. Korea},  ~~ \href{http://www.roboicorp.com/sub/develop/churo2.php}{\textcolor{gray}{\faGlobe} site}}{2020}
}
  \TitledListItem{Built a behavior recognition and indoor localization pipeline for a dog care robot, integrating YOLOv5, fine-tuned language models, and AWS-based edge streaming.}
\TitledListEnd
\vspace{1em}
\TitledListBegin{
  \TitleWithPeriod{\textbf{Film Defect Inspection Software development, UNIEYE Co. Korea} ~~ \href{http://www.unieye.co.kr/kr/product/product_view.php?idx=2}{\textcolor{gray}{\faGlobe} site}}{2019}
}
  \TitledListItem{Developed a film defect inspection system on Linux using C++, OpenCV, and Qt5, modernizing a legacy pipeline with custom UI, defect detection, and camera calibration}
\TitledListEnd
\vspace{1em}
\TitledListBegin{
  \TitleWithPeriod{\textbf{Analogue circuit Lanchpad, Hanyang University ERICA}  ~~ \href{https://www.youtube.com/watch?v=PxF3JUQmUiY}{\textcolor{red}{\faYoutube} youtube}}{2019}
}
  \TitledListItem{Built an analog launchpad circuit using flip-flops and oscillators, gaining practical experience in circuit design and hardware-level control relevant to robotics systems.}
\TitledListEnd
