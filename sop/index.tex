\documentclass[12pt]{article}

\usepackage[a4paper,margin=0.9in]{geometry}
\usepackage{setspace}
\usepackage{xcolor}
\usepackage{fancyhdr}
\usepackage{fontspec}

\setmainfont{Times New Roman}

\definecolor{sectiongray}{RGB}{60,60,60}

\pagestyle{fancy}
\fancyhf{}
\fancyhead[R]{\textcolor{gray}{ChanHui Jung}}

% 줄간격·문단
\setstretch{1.15}           % 1.25 -> 1.15
\setlength{\parskip}{0.0em} 
\setlength{\parindent}{1.5em}% 문단 첫줄 들여쓰기로 블록 간격 대신 밀도 ↑

% 살짝 더 자연스러운 줄바꿈
\emergencystretch=1.5em

\newcommand\firstName{ChanHui}
\newcommand\lastName{Jung}
\newcommand\fullName{\firstName~\lastName}

\begin{document}

\begin{center}
  {\large\bfseries Statement of Purpose}\\[-0.2em]
  {\normalsize \textcolor{sectiongray}{\fullName}}
\end{center}

How can fleets of robots navigate efficiently in crowded, dynamic environments?
This curiosity has guided my academic and professional journey in robotics.
My passion for robotics, which began in middle school, was strengthened and refined during my undergraduate years.
In my second year, I joined the AIRO Lab, where I acquired foundational research experience in electronics, perception, motor control, and system integration.

As I began to consider which area of robotics to specialize in, I joined Floatic, a robotics startup, to clarify my research direction while applying theoretical concepts to real-world applications.
I developed multi-robot navigation and fleet management systems, addressing challenges such as congestion and flow control, and I also contributed to the ROS2 and Nav2 communities, which consolidated my understanding of advanced navigation frameworks.
These experiences motivated me to pursue research in multi-agent pathfinding (MAPF) and in optimizing robot fleet management (RMF) for complex, dynamic environments.

\vspace{0.8em}
\end{document}









