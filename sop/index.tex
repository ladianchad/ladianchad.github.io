\documentclass[12pt]{article}

\usepackage[a4paper,margin=0.9in]{geometry}
\usepackage{setspace}
\usepackage{xcolor}
\usepackage{fancyhdr}
\usepackage{fontspec}

\setmainfont{Times New Roman}

\definecolor{sectiongray}{RGB}{60,60,60}

\pagestyle{fancy}
\fancyhf{}
\fancyhead[R]{\textcolor{gray}{ChanHui Jung}}

% 줄간격·문단
\setstretch{1.15}           % 1.25 -> 1.15
\setlength{\parskip}{0.0em} 
\setlength{\parindent}{1.5em}% 문단 첫줄 들여쓰기로 블록 간격 대신 밀도 ↑

% 살짝 더 자연스러운 줄바꿈
\emergencystretch=1.5em

\newcommand\firstName{ChanHui}
\newcommand\lastName{Jung}
\newcommand\fullName{\firstName~\lastName}

\begin{document}

\begin{center}
  {\large\bfseries Statement of Purpose}\\[-0.2em]
  {\normalsize \textcolor{sectiongray}{\fullName}}
\end{center}

How can humans navigate smoothly through crowded spaces with little explicit communication,while robots, each following an “optimal” path, collapse into gridlock? I confronted this paradox firsthand during my employment at Floatic, a South Korean robotics startup developing autonomous warehouse fleets for logistics and fulfillment. At a narrow warehouse intersection, the fleet I managed stalled completely, exposing how locally optimal decisions can still lead to system-level failure. As an engineer responsible for navigation and fleet coordination, I realized that autonomy must extend beyond collision avoidance to encompass global fleet movement. This realization solidified my interest in mobile robotics and multi-robot coordination, and the contrast between human fluidity and robotic stagnation continues to drive my research today.

\vspace{0.5em}
Modern industries increasingly deploy robots in unmodified, human-shared spaces among people, other robots, and vehicles. These environments expose multi-robot navigation to congestion, unpredictability, and evolving layouts. Achieving robust fleet-level autonomy under these conditions requires advanced engineering and a strong foundation in control systems, perception, and integration. My academic and professional experiences have shaped both my technical skills and my research motivation to address these challenges.

\vspace{0.5em}
My journey into robotics began in my second year at Hanyang University, when I joined the AIRO Lab. There, I built fundamentals in control, circuitry, and programming and took my first steps into research through a vision-based pose estimation project. That early exposure not only trained me in careful experimentation and reproducibility but also allowed me to mentor undergraduates, which sharpened my ability to explain complex ideas clearly. In my third year, I deepened this foundation through projects such as a LattePanda-based service robot and my senior capstone design. Working in unstructured environments with limited compute and safety constraints taught me that reliable autonomy requires more than obstacle avoidance. It demands the seamless integration of perception, navigation, and control. Leading evaluation protocols also prompted me to consider how single-robot methods scale to multi-agent systems, pushing me to move beyond brittle heuristics.

\vspace{0.5em}
At Floatic, I transitioned from observing the limits of autonomy to actively addressing them. As the engineer in charge of the navigation stack, I confronted the challenge of ensuring smooth fleet movement through narrow aisles and unpredictable human traffic. Rather than optimizing each robot in isolation, I developed decentralized navigation strategies and a supervisory coordination layer that prioritized throughput and fairness at the fleet level. This shift taught me to view navigation as a system-level challenge. In parallel, contributing to the ROS2 Navigation2 stack showed me how large-scale frameworks manage complexity and enable sustainable development. Together, these experiences reinforced my conviction that autonomy must be evaluated in terms of overall flow and resilience, not just individual robot performance.

\vspace{0.5em}
Seeking to extend these lessons into formal research, I co-authored a paper on congestion-aware multi-agent path finding (MAPF). Translating engineering insights into publishable research strengthened my identity as a researcher and deepened my commitment to bridging theory and practice. My next role at Metafarmers, a South Korean agricultural robotics company building autonomous fleets for field operations, broadened this perspective. Unlike structured warehouses, fields evolved constantly; soil shifted, crop rows deformed, and weather altered mobility. My task was to adapt navigation and coordination strategies so multiple robots could remain effective under these uncertainties. I learned that fleet-level autonomy is not only about resolving congestion but also about designing systems resilient to dynamic, evolving environments.

\vspace{0.5em}
These experiences clarified my academic path. Carnegie Mellon University (CMU)’s curriculum offers precisely the foundation I seek. Courses such as 16-833 Robot Autonomy will help me integrate perception, planning, and control, while 16-781 Planning and Decision Making in Robotics will deepen my skills in optimization and distributed decision-making. I am particularly excited about the prospect of working with Prof. Stephen Smith on multi-agent planning and Prof. Howie Choset on distributed robotics, whose research aligns closely with my industrial and academic interests. Their guidance would help me transform practical problems into rigorous research contributions.

\vspace{0.5em}
Through graduate study at CMU, I aim to develop scalable methods for fleet-level scheduling, MAPF, and planning–control interfaces, enabling multi-robot systems to operate with the same fluidity and adaptability as humans. My experiences bridging warehouse fleets, open-source frameworks, and field deployments have convinced me that true progress occurs when industrial failures are translated into academic questions, and academic solutions are returned to real-world systems. I hope to contribute to this cycle of innovation at CMU and beyond.

\end{document}









